\documentclass[12pt,a4paper]{report}
\usepackage[utf8]{inputenc}
\usepackage[T1]{fontenc}
\usepackage[francais]{babel}
\frenchbsetup{StandardLists=true} % � inclure si on utilise \usepackage[french]{babel}
\usepackage{enumitem}
\usepackage{amssymb}
\usepackage[colorlinks=true]{hyperref}
\usepackage{graphicx}
\hypersetup{urlcolor=black,linkcolor=black,citecolor=black,colorlinks=true}

\newcommand{\myTitle}{
\begingroup 
\centering 


\vspace{\stretch{1}}
{\LARGE Le support des r\'{e}seaux mobiles dans IPv6}% Title

\vspace*{1\baselineskip}

\LARGE % Small caps
Le protocole NEMO
\vspace{\stretch{2}}

\vspace{\stretch{1}}
{\Large SARI Soumia \& Kaoutar Sarah SANFILIPPO\par} 
\vspace{\stretch{2}}

\vspace{\stretch{1}}
{\large 2014 - 2015}
\vspace{\stretch{2}}%pour g�rer les espaces sur des pages,  \stretch{} fait des espaces proportionnels
\setcounter{page}{0}

\endgroup}


\begin{document}
\thispagestyle{empty}
\myTitle

\renewcommand{\contentsname}{Sommaire}
\tableofcontents

\newpage
\section{Introduction}

L'explosion des technologies de communication sans fil (e.g. Wi-Fi) a fait \'{e}merger un nouveau
concept dans les r\'{e}seaux IP : la mobilit\'{e}. Lorsqu'un utilisateur b\'{e}n\'{e}ficie d'une connexion sans fil \`a
l'Internet, celui-ci peut se d\'{e}placer tout en communiquant. Cependant, de tels d\'{e}placements
requi\'{e}rent un support sp\'{e}cifique au niveau de la couche 3 du mod\`{e}le TCP/IP, sans lequel toutes les
communications seront rompues lors d'un changement de sous-r\'{e}seaux IPv6. Pour palier ces
probl\`{e}mes, l'organisme de standardisation IETF a d\'{e}fini le protocole NEMO (Network Mobility) Basic
Support qui place la gestion de la mobilit\'{e} au niveau des routeurs, ce qui permet le mouvement de
r\'{e}seaux entiers tout en conservant la complexit\'{e} de la gestion des d\'{e}placements dans l'Internet sur
ces dits routeurs.\cite{ref}

 
\section{Les R\'eseaux NEMO}
Un r\'eseau mobile ou r\'eseau NEMO peut \^etre d\'efini comme un r\'eseau ou sous r\'eseau en d\'eplacement, connect\'e \`a Internet par l'interm\'ediaire d'un ou plusieurs routeurs qui changent leurs points d'attachements dans la topologie Internet. La gestion de la mobilit\'e des r\'eseaux NEMO doit assurer d'une mani\`ere transparente la continuit\'e des services Internet  pour les stations ou terminaux embarqu\'es.

\section{Le Protocol Nemo  Basic Support (BS)}
Le protocole NEMO BS est une extension de Mobile IPv6(MIPV6) pour supporter la mobilit\'{e} d'un r\'{e}seau entier (r\'{e}seau NEMO) qui change son point d'attachement \`{a} Internet. NEMO BS  assure d'une mani\`ere transparente la continuit\'e des sessions ouvertes pour tous les n�uds dans le r\'eseau mobile NEMO.

\newpage
\section{Terminologie}
\begin{itemize}[label=$\square$]
\item  Mobile Node (MN) ou n\oe{}ud mobile: un terminal mobile ou un routeur mobile qui change son point d'attachement d'un r\'eseau \`a un autre.
\item  Home Network ou r\'eseau m\`ere: c'est le r\'eseau auquel est attach\'e initialement un MN.
\item  Home Agent (HA) ou agent m\`ere : un routeur d'acc\`es particulier situ\'e dans le r\'eseau m\`ere qui participe \`a la gestion de la mobilit\'e du MN.
\item  Foreign Network ou r\'eseau visit\'e: n'importe quel r\'eseau autre que le r\'eseau m\`ere auquel le MN est connect\'e.
\item  Foreign Agent (FA) ou agent visit\'e: un routeur d'acc\`es dans le r\'eseau visit\'e qui fournit au MN un service de routage des paquets qui lui sont destin\'es par le HA.
\item  Home Address (HoA) ou adresse m\`ere : est l'adresse IP du MN sur son r\'eseau m\`ere. Elle fournit l'identification du MN pour tous ses correspondants.
\item  Care-of Address (CoA) ou adresse temporaire : est l'adresse IP de localisation du MN, obtenue au r\'eseau visit\'e, et qui lui permet  d'envoyer et recevoir des paquets sur ce r\'eseau. 
\item  Correspondant Node (CN) ou n�ud correspondant : est un terminal en communication  avec le MN. Un CN peut \^etre fixe ou mobile.
\item MNNs Mobile Network Node : Noeud Mobile R\'eseau.
\item MNP Mobile Network Prefix :R\'eseau mobile pr\'efixe.
\item MR Mobile Router : le routeur mobile.


\end{itemize}

\newpage
\section{Fonctionnement  de base du protocole NEMO BS}

Le standard NEMO, pour Network MObility, de l'IETF est d\'eriv\'e de Mobile IP pour G\'erer la mobilit\'e des r\'eseaux IP . Un sous-r\'eseau comporte des n\oe{}uds mobiles avec le m\^eme pr\'efixe d'adresse IP (MNP). Ce sous-r\'eseau est associ\'e \`a un
 r\'eseau m\`ere et peut changer de r\'eseau en changeant de point d'ancrage, c'est-\`a-dire de point d'acc\`es. Un des objectifs est de ne pas imposer des modifications aux n\oe{}uds mobiles.
Deux nouvelles entit\'es sont introduites : le n\oe{}ud du r\'eseau mobile (MNN) et le routeur mobile (MR). Le MNN est un n\oe{}ud mobile qui fait partie du r\'eseau mobile. Le routeur mobile est l'entit\'e la plus importante de NEMO. Le changement de point d'acc\`es ne provoque pas de changement d'adresse IP du MNN. La gestion de la mobilit\'e est d\'el\'egu\'ee au MR. Le MR, comme le n\oe{}ud mobile de Mobile IP, poss\`ede deux adresses. Le premi\`ere, l'adresse m\`ere, est permanente et identifie le MR dans le r\'eseau m\`ere. Son pr\'efixe est le m\^eme que les MNN. Du fait qu'elle ne change pas, tout correspondant sur Internet peut atteindre le MR. La seconde adresse, l'adresse temporaire, est obtenue dans le r\'eseau visit\'e, o\`u se trouve le point d�acc\`es. Le protocole NEMO \'etablit ainsi une association entre le pr\'efixe MNP du sous-r\'eseau mobile et l'adresse temporaire. Lorsque le r\'eseau change de point d'ancrage, le MNP envoie son adresse temporaire \`a l'agent m\`ere (HA) du r\'eseau m\`ere. L'agent m\`ere actualise l'association entre le pr\'efixe du sous-r\'eseau et l'adresse temporaire. Un tunnel est ensuite \'etabli entre le MR et le HA pour transmettre les paquets provenant d'Internet vers le MR. Le HA encapsule le paquet pour le transmettre au MR, puis le MR d\'esencapsule le paquet. Le MR utilise ensuite le protocole de routage du sous-r\'eseau pour transmettre les paquets vers
le MNN . Pour le correspondant, la mobilit\'e est alors transparente, puisqu'il envoie ses donn\'ees \`a
un routeur du r\'eseau m\`ere. Les paquets \`a destination d'un correspondant sur Internet sont envoy\'es au MR, avec le protocole de routage utilis\'e par le sous-r\'eseau mobile, puis encapsul\'es
pour \^etre envoy\'es au HA. Le HA d\'esencapsule le paquet, puis transmet le paquet au correspondant. Tout le trafic en provenance et \`a destination du sous-r\'eseau mobile passe alors par le HA, ce qui rend le routage sous optimal.`  
\cite{benaouda2014etude}

\begin{figure}[h]
\begin{center}
\includegraphics[scale=0.5]{img8}
\caption[9pt]{Fig.1 Architecture}
\end{center}
\end{figure}

\newpage
\section{Sc\'{e}nario}

Apr\`{e}s avoir d\'{e}marr\'{e}, le routeur mobile se configure automatiquement pour assurer une connectivit\'{e} aux utilisateurs associ\'{e}s. Ces derniers vont pouvoir automatiquement d\'{e}couvrir des services IPv6 fournis par l'op\'{e}rateur. Enfin, le routeur en mouvement passant d'un r\'{e}seau d'acc\`{e}s � un autre conserve les connexions r\'{e}seaux de mani\`{e}re transparente pour l'utilisateur.
\begin{figure}[h]
\begin{center}
\includegraphics[scale=0.5]{R2}
\caption[9pt]{Fig.2 Exemple de r\'{e}seau mobile embarquant un routeur mobile munis de multiples interfaces. Le routeur mobile assure la continuit\'{e} de service tout au long des d\'{e}placements du train.}
\end{center}
\end{figure}

\newpage
\paragraph{Gestion de la Mobilit\'{e}:} 
Le routeur mobile op\`{e}re le protocole NEMO BS qui lui permet d'\^{e}tre toujours joignable par l'interm\'{e}diaire de son adresse principale tout comme les clients associ\'{e}s dans le r\'{e}seau mobile. Cette adresse principale est associ\'{e}e \'{a} une adresse temporaire aupr\`{e}s d'un \'{e}quipement appel\'{e} agent m\`{e}re. Cette adresse temporaire repr\'{e}sente la position r\'{e}elle du routeur mobile dans la topologie d'Internet et est mise-\'{a}-jour \'{a} chaque d\'{e}placement du r\'{e}seau mobile vers un nouveau r\'{e}seau d'acc\'{e}s.
L'ensemble des flux \'{a} destination du r\'{e}seau mobile passent toujours par l'agent m\`{e}re, qui peut donc assurer la continuit\'{e} des flux tout au long des d\'{e}placements du r\'{e}seau mobile.

\paragraph{Multi-domiciliation:} 
Le routeur mobile dispose de plusieurs interface r\'{e}seau lui permettant de se connecter en parall\`{e}le a plusieurs r\'{e}seaux d'acc\`{e}s IPv6. Son adresse principale est alors associ\'{e}e \`{a} plusieurs adresses IPv6 temporaires (une par interface) gr\^{a}ce au protocole Multiple Care-of Addresses registration (MCoA). Plusieurs chemins concurrents peuvent ainsi \^{e}tre maintenus entre le routeur mobile et son agent m\`{e}re.
Les flux de l'Internet \'{a} destination du r\'{e}seau mobile ou inversement font l'objet d'une d\'{e}cision de routage respectivement sur l'agent m\`{e}re ou le routeur mobile. Ces d\'{e}cisions sont prises en fonction de \'{e}\'{e}rences ou politiques de routages pr\'{e}sentes sur chacune de ces entit\'{e}s.
Les flux peuvent ainsi \^{e}tre partag\'{e}s entre diff\'{e}rent chemins selon leur protocol et/ou port. Le routeur mobile et l'agent m\`{e}re peuvent \'{e}galement plus facilement faire face \'{a} une panne ou une d\'{e}connexion de l'un des r\'{e}seaux d'acc\`{e}s en redirigeant les flux vers les interfaces disponibles.
Nous travaillons \'{e}galement \'{a} la gestion de routeurs mobiles multiples au sein d'un m\^{e}me r\'{e}seau mobile. Nous nous int\'{e}ressons notamment aux m\'{e}canismes de redondance des routeurs mobiles tout en \'{e}tendant la mise en oeuvre du partage de charge et de tol\'{e}rance au fautes dans ce contexte. \cite{ref3}


\section{Contraintes}
La gestion de la mobilit\'e des r\'eseaux mobiles doit se faire face \`a de nombreuse contraintes:
\begin{enumerate}
  \item il convient de supporter les r\'eseaux mobiles en nombre et en taille importantes , en consid\'erant divers types de configuration ( un seul sous-r\'eseau , la multi-domiciliation, la mobilit\'e enchain\'ee).
  \item Le nombre \'elev\'e de correspondants nous impose de minimiser la quantit\'e de messages de controle relatifs \`a la gestion de la mobilit\'e tout en optimisant le routage.
  \item Ces messages doivent \^etre \'echang\'es en toute s\'ecurit\'e et authentifi\'es par leurs destinataires pour s'assur\'er qu'ils ne sont pas envoy\'es par un usurpateur. \cite{referance}
\end{enumerate}

\section{Conclusion}
Le support des r\'eseaux mobiles est \`a pr\'esent un sujet qui int\'eresse se nombreux industriels, allant des fournisseurs
 d'\'equipement r\'eseaux ou d'\'electronique grand publique jusq'aux fabriquant d'automobiles, en passant par les op\'erateurs de t\'elephone et de transport public. 

\bibliographystyle{unsrt} %plain, unsrt, alpha, abbrv, acm, apalike
\bibliography{nemo} % mon fichier de base de donn�es s'appelle bibli.bib

\end{document} 
