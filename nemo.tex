\documentclass[12pt]{article}
\usepackage[latin1]{inputenc}
\usepackage[T1]{fontenc}
\usepackage[francais]{babel}
\usepackage[colorlinks=true]{hyperref}
\usepackage{graphicx}
\hypersetup{urlcolor=blue,linkcolor=black,citecolor=black,colorlinks=true}

\title{Le support des r\'{e}seaux mobiles dans IPv6
           le protocole NEMO }
\author{SANFILIPPO Kaoutar Sarah , SARI Soumia}
\date{2014 - 2015}

\begin{document}

\maketitle
\newpage
\renewcommand{\contentsname}{Sommaire}
\tableofcontents

\newpage
\section{Introduction}

L'explosion des technologies de communication sans fil (e.g. Wi-Fi) a fait \'{e}merger un nouveau
concept dans les r\'{e}seaux IP : la mobilit\'{e}. Lorsqu'un utilisateur b\'{e}n\'{e}ficie d'une connexion sans fil \`a
l'Internet, celui-ci peut se d\'{e}placer tout en communiquant. Cependant, de tels d\'{e}placements
requi\'{e}rent un support sp\'{e}cifique au niveau de la couche 3 du mod\`{e}le TCP/IP, sans lequel toutes les
communications seront rompues lors d'un changement de sous-r\'{e}seaux IPv6. Pour palier ces
probl\`{e}mes, l'organisme de standardisation IETF a d\'{e}fini le protocole NEMO (Network Mobility) Basic
Support qui place la gestion de la mobilit\'{e} au niveau des routeurs, ce qui permet le mouvement de
r\'{e}seaux entiers tout en conservant la complexit\'{e} de la gestion des d\'{e}placements dans l'Internet sur
ces dits routeurs.

\newpage
\section{Sc\'{e}nario}

Apr\`{e}s avoir d\'{e}marr\'{e}, le routeur mobile se configure automatiquement pour assurer une connectivit\'{e} aux utilisateurs associ\'{e}s. Ces derniers vont pouvoir automatiquement d\'{e}couvrir des services IPv6 fournis par l'op\'{e}rateur. Enfin, le routeur en mouvement passant d'un r\'{e}seau d'acc\`{e}s à un autre conserve les connexions r\'{e}seaux de mani\`{e}re transparente pour l'utilisateur.

\begin{figure}[h]
\begin{center}
\includegraphics[scale=0.7]{R2.png}
\caption[9pt]{Fig.1 Exemple de r\'{e}seau mobile embarquant un routeur mobile munis de multiples interfaces. Le routeur mobile assure la continuit\'{e} de service tout au long des d\'{e}placements du train.}
\end{center}
\end{figure}

\newpage
\paragraph{Gestion de la Mobilit\'{e}:} 
Le routeur mobile op\`{e}re le protocole NEMO BS qui lui permet d'\^{e}tre toujours joignable par l'interm\'{e}diaire de son adresse principale tout comme les clients associ\'{e}s dans le r\'{e}seau mobile. Cette adresse principale est associ\'{e}e \'{a} une adresse temporaire aupr\`{e}s d'un \'{e}quipement appel\'{e} agent m\`{e}re. Cette adresse temporaire repr\'{e}sente la position r\'{e}elle du routeur mobile dans la topologie d'Internet et est mise-\'{a}-jour \'{a} chaque d\'{e}placement du r\'{e}seau mobile vers un nouveau r\'{e}seau d'acc\'{e}s.
L'ensemble des flux \'{a} destination du r\'{e}seau mobile passent toujours par l'agent m\`{e}re, qui peut donc assurer la continuit\'{e} des flux tout au long des d\'{e}placements du r\'{e}seau mobile.

\paragraph{Multi-domiciliation:} 
Le routeur mobile dispose de plusieurs interface r\'{e}seau lui permettant de se connecter en parall\`{e}le a plusieurs r\'{e}seaux d'acc\`{e}s IPv6. Son adresse principale est alors associ\'{e}e \`{a} plusieurs adresses IPv6 temporaires (une par interface) gr\^{a}ce au protocole Multiple Care-of Addresses registration (MCoA). Plusieurs chemins concurrents peuvent ainsi \^{e}tre maintenus entre le routeur mobile et son agent m\`{e}re.
Les flux de l'Internet \'{a} destination du r\'{e}seau mobile ou inversement font l'objet d'une d\'{e}cision de routage respectivement sur l'agent m\`{e}re ou le routeur mobile. Ces d\'{e}cisions sont prises en fonction de \'{e}\'{e}rences ou politiques de routages pr\'{e}sentes sur chacune de ces entit\'{e}s.
Les flux peuvent ainsi \^{e}tre partag\'{e}s entre diff\'{e}rent chemins selon leur protocol et/ou port. Le routeur mobile et l'agent m\`{e}re peuvent \'{e}galement plus facilement faire face \'{a} une panne ou une d\'{e}connexion de l'un des r\'{e}seaux d'acc\`{e}s en redirigeant les flux vers les interfaces disponibles.
Nous travaillons \'{e}galement \'{a} la gestion de routeurs mobiles multiples au sein d'un m\^{e}me r\'{e}seau mobile. Nous nous int\'{e}ressons notamment aux m\'{e}canismes de redondance des routeurs mobiles tout en \'{e}tendant la mise en oeuvre du partage de charge et de tol\'{e}rance au fautes dans ce contexte.






\section{Usages}
\section{Contraintes}
\section{Conclusion}

\end{document} 
