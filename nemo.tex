\documentclass[a4paper]{article}
\usepackage[latin1]{inputenc}
\usepackage[T1]{fontenc}
\usepackage[francais]{babel}
\usepackage[colorlinks=true]{hyperref}
\hypersetup{urlcolor=blue,linkcolor=black,citecolor=black,colorlinks=true}

\title{Le support des r\'{e}seaux mobiles dans IPv6
           le protocole NEMO }
\author{SANFILIPPO Kaoutar Sarah , SARI Soumia}
\date{2014 - 2015}

\begin{document}

\maketitle
\newpage
\renewcommand{\contentsname}{Sommaire}
\tableofcontents

\newpage
\section{Introduction}

L'explosion des technologies de communication sans fil (e.g. Wi-Fi) a fait \'{e}merger un nouveau
concept dans les r\'{e}seaux IP : la mobilit\'{e}. Lorsqu'un utilisateur b\'{e}n\'{e}ficie d'une connexion sans fil \`a
l'Internet, celui-ci peut se d\'{e}placer tout en communiquant. Cependant, de tels d\'{e}placements
requi\'{e}rent un support sp\'{e}cifique au niveau de la couche 3 du mod\`{e}le TCP/IP, sans lequel toutes les
communications seront rompues lors d'un changement de sous-r\'{e}seaux IPv6. Pour palier ces
probl\`{e}mes, l'organisme de standardisation IETF a d\'{e}fini le protocole NEMO (Network Mobility) Basic
Support qui place la gestion de la mobilit\'{e} au niveau des routeurs, ce qui permet le mouvement de
r\'{e}seaux entiers tout en conservant la complexit\'{e} de la gestion des d\'{e}placements dans l'Internet sur
ces dits routeurs.

\section{Usages}
\section{Contraintes}
\section{Conclusion}

\end{document} 
